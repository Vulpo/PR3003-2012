\documentclass[10pt,a4paper]{article}
\usepackage[utf8]{inputenc}
\usepackage[francais]{babel}
\usepackage[T1]{fontenc}
\usepackage{amsmath}
\usepackage{amsfonts}
\usepackage{amssymb}
\usepackage{makeidx}
\usepackage{graphicx}
\usepackage{float} %pour que l'option "H" fonctionne dans figure.
\usepackage[left=2cm,right=2cm,top=2cm,bottom=2cm]{geometry}
\usepackage{hyperref} % lien hypertext
\usepackage{tikz,tkz-tab} %tableau de variations
%\hypersetup{pdfborder={0 0 0}}

\begin{document}
%\setcounter{page}{1} % enlever numero de page


\begin{minipage}{0.5\linewidth} % permet de faire les deux colonnes pour mettre l'image à  droite du texte
Colpier Clément\\
Fornara Thibault\\
Pellegrino Guillaume\\
Renard Charles\\



26/01/13
\end{minipage}
\begin{minipage}{0.5\linewidth}
\begin{flushright}
\includegraphics[scale=0.2]{logo-esiee.jpg}

\end{flushright}
\end{minipage}


\vspace{8cm}

\begin{center}
\LARGE Projet de Mathématiques appliquées \\
\-
\LARGE PR3003
\\

\end{center}




\newpage
-
\newpage
\tableofcontents              % Table des matieres
\clearpage


\section{Déterminer l'équation différentielle vérifiée par M(t)=(x(t),y(t)).}
\begin{figure}[H]
	\includegraphics[scale=0.7]{GraphMath2.png}
	%\caption{Histoire de Sanofi}
\end{figure}
La masselotte M se déplace uniquement selon la composante tangentielle. Pour déterminer l'équation différentielle on va donc particulièrement s'intéresser à l'équation sur la composante tangentielle.\\
Pour cela, on commence à faire la somme des forces s'exerçant sur la composante tangentielle $\vec{u_t}$ et normale $\vec{u_n}$ selon la seconde loi de Newton (PFD) :
\[
   \left \{
   \begin{array}{r c l}
      P_t+T_t=ma_t  \\
      P_n+R_n+T_n=0 
   \end{array}
   \right .
\]
On s'intéresse à l'équation: \[\fbox{$P_t+T_t=ma_t$}\] \\ 
Pour déterminer l'équation différentielle, on doit alors projeter $\vec{T}$ et  $\vec{mg}$ sur $\vec{u_t}$. \\
On projette $\vec{mg}=-mg.\vec{u_y}$ sur $\vec{u_t}$\\

\subsection{Projection du Poids sur la composante tangentielle}
\begin{figure}[H]
	\includegraphics[scale=0.7]{GraphMathZoomProjectionPoids.png}
	%\caption{Histoire de Sanofi}
\end{figure}

On remarque sur le graphique que $P_t=P.\cos(\alpha)$\\
On cherche à déterminer $\alpha$.
On calcule la pente a de la tige parabolique. $a=\frac{\partial y}{\partial x}=\frac{\partial x^2/2}{\partial x}=x$ \\
En $M(x_0,y_0)$ la pente a de la tige parabolique vaut donc $x_0$.
Cette pente a nous permet de calculer l'angle $\alpha$. En effet, on remarque graphiquement que $\tan(\alpha)=\frac{1}{a}$. On en déduit: $\alpha=\tan^{-1}(\frac{1}{x_0})$\\
Au final on trouve donc: $P_t=P.\cos(\tan^{-1}(\frac{1}{x_0})) $\\
Or $\cos(\tan^{-1}(x))=\frac{1}{\sqrt{1+x^2}}$  On en déduit donc: $P_t=P.\frac{1}{\sqrt{1+1/x^2_0}}$ D'où: 
\[\fbox{$P_t=P.\frac{x_0}{\sqrt{1+x^2_0}}$}\]

\subsection{Projection de la tension du ressort su la composante tangentielle}
On projette désormais $\vec{T}$ sur $\vec{u_t}$.\\
\subsubsection{Methode de Guillaume, diff d'angle}
\begin{figure}[H]
	\includegraphics[scale=0.35]{GraphMathZoomProjectionT2.png}
	%\caption{Histoire de Sanofi}
\end{figure} 
$\cos(\phi)=\frac{x}{\sqrt{x^2+y^2}}=\frac{1}{\sqrt{1+x^2}}$ \\

$\cos(\theta)=\frac{x}{\sqrt{(1-y)^2+x^2}}=\frac{x}{\sqrt{1+x^4/4}}$ \\
Et: 
$T_t=T.\cos(\alpha 2)=T.\cos(\phi - \theta)=T[\cos(\phi).\cos(\theta)+\sin(\phi).\sin(\theta)]$\\
On en déduit:\\
$T_t=T[\frac{1}{\sqrt{1+x^2}}.\frac{x}{\sqrt{1+x^4/4}} + \sin(\cos^{-1}(\frac{1}{\sqrt{1+x^2}})).\sin(\cos^{-1}(\frac{x}{\sqrt{1+x^4/4}}))]$\\
Or: $\sin(\cos^{-1}(u))=\sqrt{1-u^2}$\\
On trouve donc:\[\fbox{$T_t=T.[\frac{1}{\sqrt{1+x^2}}.\frac{x}{\sqrt{1+x^4/4}} + \sqrt{1-\frac{1}{1+x^2}}.\sqrt{1-\frac{x^2}{1+x^4/4}}]$}\]\\

%La pente de la tangente vaut $x_0$. Celle de $\vec{PM}$ vaut $\frac{y_0-1}{x_0}$.\\
%On en déduit ainsi: $\tan(\phi)=x_0$ et $\tan(\theta)=\frac{y_0-1}{x_0}$
%On obtient ainsi $\alpha=\tan^{-1}(x_0) - \tan^{-1}(\frac{y_0-1}{x_0})$ et on en déduit: $T_t=T.\cos(\tan^{-1}(x_0) - \tan^{-1}(\frac{y_0-1}{x_0}))$ \\

\subsubsection{Methode de Charles, Al-Kashi}
\begin{figure}[H]
	\includegraphics[scale=0.5]{al-kashi.png}
	%\caption{Histoire de Sanofi}
\end{figure}
On note x,y les coordonnées du point M.\\
$a=\sqrt{(x_M-x_P)^2+(y_M-y_P)^2}=\sqrt{x^2+(\frac{x^2}{2}-1)}=\sqrt{x^2+\frac{x^4}{4}-x^2+1}=\sqrt{\frac{x^4}{4}+1}$\\
$b=\sqrt{(x_M-x_P)^2+(y_M-\Delta(0))^2}=\sqrt{x^2+(\frac{x^2}{2}+\frac{x^2}{2})^2}=\sqrt{x^2+x^4}=x\sqrt{1+x^2}$\\
note : faut-il mettre plutôt $|x|\sqrt{1+x^2}$ ?\\
$c=\sqrt{(y_p-\Delta(0))^2}=\sqrt{(1+\frac{x^2}{2})^2}=1+\frac{x^2}{2}$\\
D'après le théorème d'Al-Kashi : 
$x_1 = \frac{5 + \sqrt{25 - 4 \times 6}}{2} = 3$\\
$c^2 = a^2 + b^2 - 2ab\times\cos(\beta)$\\
$\cos(\beta)=\frac{a^2+b^2-c^2}{2ab}=\frac{x^4/4+1+x^2+x^4-1-x^2-x^4/4}{2x\sqrt{x^4/4+1}\sqrt{1+x^2}}=\frac{x^3}{2\sqrt{\frac{x^4}{4}+1}\sqrt{1+x^2}}$\\
On trouve donc :
\[\fbox{ $T_t=T\times\frac{x^3}{2\sqrt{\frac{x^4}{4}+1}\sqrt{1+x^2}}$ }\]\\

\subsection{Determination de $ \|T\| $}
On détermine la valeur de la tension du ressort.\\
$T=k(l-l_0)=k(\sqrt{(x_M-x_P)^2+(y_M-y_P)^2}-l_0)=k(\sqrt{x^2+(x^2/2-1)^2}-l_0)$\\
$T=k(\sqrt{x^2+x^4/4-x^4+1}-l_0)$\\
\[\fbox{ $T=k(\sqrt{1+\frac{x^4}{4}}-l_0)$ }\]

\subsection{Détermination de $a_t$}
On a vu dans la première équation que $a_n=0$. On en déduit: $||\vec{a}||=a_t$\\
Avec une accélération normale nulle, on peut écrire la formule de l'accélération dans le repère de Frenet ainsi : $a_t=||\vec{a}||$\\ %=$\sqrt{a_x^2+a_y^2}$
Or $||\vec{a}||=\frac{\partial v}{\partial t}=\frac{\partial \pm\sqrt{\dot{x}^2+\dot{y}^2}}{\partial t}$\\
$ \dot{y}=\frac{\partial y}{\partial t}=\frac{\partial y}{\partial x}\times\frac{\partial x}{\partial t}=x\dot{x} $\\
$ v=\sqrt{\dot{x}^2+\dot{x}^2x^2}=\dot{x}\sqrt{1+x^2} $\\
$ \frac{\partial v}{\partial t}=\ddot{x}\sqrt{1+x^2}+\frac{\dot{x}^2x}{\sqrt{1+x^2}} $\\
On trouve: \[\fbox{$a_t=\ddot{x}.\sqrt{1+x^2}+ \frac{\dot{x}^2.x}{\sqrt{1+x^2}}$}\] (Equation de Charles)
%On obtient alors: $a_t=\sqrt{\ddot{x}^2+\ddot{y}^2}$

\subsection{Détermination de l'équation différentielle}
A l'aide de ce qu'on a calculé précédemment on développe l'équation $mg_t+T_t=ma_t$ pour déterminer l'équation différentielle.
On obtient alors: 
\subsubsection{Equa diff de Guillaume}
En développant et en prenant k=m, g=1 et $a=l_0$ (données de l'énoncé), on obtient :\\
$m.1.\frac{x}{\sqrt{1+x^2}} + m(\sqrt{x^4/4+1}-a).[\frac{1}{\sqrt{1+x^2}}.\frac{x}{\sqrt{1+x^4/4}} + \sqrt{1-\frac{1}{1+x^2}}.\sqrt{1-\frac{x^2}{1+x^4/4}}] - m.\ddot{x}.\sqrt{1+x^2} - m.\frac{\dot{x}^2.x}{\sqrt{1+x^2}} = 0$ \\
 
$\frac{x}{\sqrt{1+x^2}} + (\sqrt{x^4/4+1}-a).[\frac{1}{\sqrt{1+x^2}}.\frac{x}{\sqrt{1+x^4/4}} + \sqrt{1-\frac{1}{1+x^2}}.\sqrt{1-\frac{x^2}{1+x^4/4}}] - \ddot{x}.\sqrt{1+x^2} - \frac{\dot{x}^2.x}{\sqrt{1+x^2}} = 0$
  
$\frac{x}{\sqrt{1+x^2}} + (\sqrt{x^4/4+1}-a).[\frac{1}{\sqrt{1+x^2}}.\frac{x}{\sqrt{1+x^4/4}} + \frac{x}{\sqrt{1+x^2}}.\sqrt{\frac{x^4/4-x^2+1}{1+x^4/4}}] - \ddot{x}.\sqrt{1+x^2} - \frac{\dot{x}^2.x}{\sqrt{1+x^2}} = 0$
  
$\frac{x}{\sqrt{1+x^2}} + [\frac{x}{\sqrt{1+x^2}} + \frac{x.\sqrt{x^4/4-x^2+1}}{\sqrt{1+x^2}}] -a.[\frac{1}{\sqrt{1+x^2}}.\frac{x}{\sqrt{1+x^4/4}} + \frac{x}{\sqrt{1+x^2}}.\sqrt{\frac{x^4/4-x^2+1}{1+x^4/4}}] - \ddot{x}.\sqrt{1+x^2} - \frac{\dot{x}^2.x}{\sqrt{1+x^2}} = 0$
 
$\frac{x}{1+x^2} + [\frac{x}{1+x^2} + \frac{x.\sqrt{x^4/4-x^2+1}}{1+x^2}] -a.[\frac{x}{(1+x^2).\sqrt{1+x^4/4}} +\frac{x.\sqrt{x^4/4-x^2+1}}{(1+x^2).\sqrt{1+x^4/4}}] - \ddot{x} - \frac{\dot{x}^2.x}{1+x^2} = 0$

$\frac{2x+x.\sqrt{x^4/4-x^2+1}-\dot{x}^2.x}{1+x^2} -a.\frac{x+x.\sqrt{x^4/4-x^2+1}}{(1+x^2).\sqrt{1+x^4/4}} - \ddot{x}= 0$

$-\frac{2x+x.\sqrt{x^4/4-x^2+1}-\dot{x}^2.x}{1+x^2} +a.\frac{x+x.\sqrt{x^4/4-x^2+1}}{(1+x^2).\sqrt{1+x^4/4}} + \ddot{x}= 0$

$-\frac{2x+x.\sqrt{(x^2/2-1)^2}-\dot{x}^2.x}{1+x^2} +a.\frac{x+x.\sqrt{(x^2/2-1)^2}}{(1+x^2).\sqrt{1+x^4/4}} + \ddot{x}= 0$

\[ \fbox{$\frac{-x^3/2-x+\dot{x}^2x}{1+x^2} + \frac{a.x^3}{2(1+x^2).\sqrt{1+x^4/4}} + \ddot{x}= 0$} \]
  
  
% \[mg.\cos(\tan^{-1}(x_0)) + k(l-l_0).\cos(\tan^{-1}(x_0) - \tan^{-1}(\frac{y_0-1}{x_0}))=\sqrt{\ddot{x}^2+\ddot{y}^2}\]
 %En développant on a:\[mg.\cos(\tan^{-1}(x)) + k(\sqrt{(x^2/2-1)^2+x^2}-l_0).\cos(\tan^{-1}(x) - \tan^{-1}(\frac{x^2/2-1}{x})) - \sqrt{\ddot{x}^2+1}=0\]

\subsubsection{Equa diff de Charles}
On calcule maintenant l'équatio ndifférentielle du système en s'aidant des résultats précédents.\\
On part de l'équation $ P_t+T_t=ma_t $\\
avec $T_t=k(l-l_0)$ et $P_t=-mg$\\
En développant les expression on obtient :\\
$k(\sqrt{x^4/4+1}-l_0)\times \frac{x^3}{2\sqrt{x^4/4+1}\sqrt{1+x^2}}-mg\frac{x}{\sqrt{1+x^2}}=m(\ddot{x}\sqrt{1+x^2}+\frac{\ddot{x}^2x}{\sqrt{1+x^2}})$\\
$ \frac{k}{m}(\frac{x^3}{2\sqrt{1+x^2}}-\frac{x^3\times l_0}{2\sqrt{x^4/4+1}\sqrt{1+x^2}})-\frac{xg}{\sqrt{1+x^2}}-\ddot{x}\sqrt{1+x^2}-\frac{\dot{x}^2x}{\sqrt{1+x^2}}=0 $\\
En prenant k=m, g=1 et $a=l_0$ (données de l'énoncé), on obtient :\\
$ -\frac{x^3}{2\sqrt{1+x^2}}+\frac{x^3\times a}{2\sqrt{x^4/4+1}\sqrt{1+x^2}}+\frac{x}{\sqrt{1+x^2}}+\ddot{x}\sqrt{1+x^2}+\frac{\dot{x}^2x}{\sqrt{1+x^2}}=0 $\\
$ (-\frac{x^3}{2}+x)\frac{1}{2\sqrt{1+x^2}}+\frac{x^3\times a}{2\sqrt{x^4/4+1}\sqrt{1+x^2}}+\ddot{x}\sqrt{1+x^2}+\frac{\dot{x}^2x}{\sqrt{1+x^2}}=0 $\\
$ (-\frac{x^3}{2}+x)\frac{1}{2\sqrt{1+x^2}}+\frac{x^3\times a}{2\sqrt{x^4/4+1}\sqrt{1+x^2}}+\ddot{x}\sqrt{1+x^2}+\frac{\dot{x}^2x}{\sqrt{1+x^2}}=0 $\\
$ (-\frac{x^3}{2}+x)\frac{1}{1+x^2}+\frac{x^3\times a}{2\sqrt{x^4/4+1}(1+x^2)}+\ddot{x}+\frac{\dot{x}^2x}{\sqrt{1+x^2}}=0 $\\
\[ \fbox{$ \ddot{x}+\frac{\dot{x}^2x-x^3/2-x}{1+x^2}+\frac{x^3\times a}{2\sqrt{x^4/4+1}(1+x^2)}=0 $}\]

\section{Dans toute la suite on supposera que g=1, k=m et on notera $a=l_0$ et on s'intéressera particulièrement par l'équation vérifié par x(t).}
\subsection{Montrer que l'équation est de la forme: $\ddot{x} + f(x,\dot{x},a) = 0.$}
On a bien $\ddot{x} + f(x,\dot{x},a) = 0.$ avec $f(x,\dot{x},a)=\frac{\dot{x}^2x+x^3/2+x}{1+x^2}-\frac{x^3\times a}{2\sqrt{x^4/4+1}(1+x^2)}$ 
\subsection{Détermination des points d'équilibre}

Les points d'équilibre sont les points où la vitesse du système et donc de la masselotte est nulle. Ainsi, les tèrmes lié à la vitesse et à l'accélération du système sont nuls.\\
Les points d'équilibres correspondent aux solutions de l'équation :\\
$ \ddot{x}+\frac{\dot{x}^2x-x^3/2-x}{1+x^2}+\frac{x^3\times a}{2\sqrt{x^4/4+1}(1+x^2)}=0 $
où 
%$\ddot{x}=\frac{\dot{x}^2x-x^3/2-x}{1+x^2}=0$\\
$\ddot{x}=0$ et $\dot{x}=0$
Ainsi on a :\\
$\frac{-x^3/2-x}{1+x^2} + \frac{x^3\times a}{2\sqrt{x^4/4+1}(1+x^2)}=0$ \\
Une première solution correspond à $x_0=0$ et $y_0=0$ \\
En multipliant de part et d'autre de l'équation par $\frac{2(1+x^2)}{x}$ :\\
$x^2-\frac{x^2\times l_0}{\sqrt{1+x^4/4}}+2=0$\\
$\sqrt{1+x^\frac{4}{4}}(x^2+2)=x^2\times l_0$\\
$\sqrt{1+y^2}(2y+2)=2yl_0$\\
$(1+y^2)(4y^2+8y+4)=4y^2\times l_0^2$\\
$(1+y^2)(y^2+2y+1)=y^2\times l_0^2$\\
$(\frac{1}{y}+y)(y+2+\frac{1}{y})=l_0^2$\\
On pose $X=y+\frac{1}{y}$ ($y\neq 0$):\\
$X(X+2)=l_0^2$\\
$X^2+2X-l_0^2=0$\\
Dont on calcule le déterminant :
$\Delta_X=4(1+l_0^2)$\\
D'où les solutions intermédiaires :
$X_1=-1-\sqrt{1+l_0^2}$\\
$X_2=-1+\sqrt{1+l_0^2}$\\
$y+\frac{1}{y}=X$\\
$y^2+1=Xy$\\
$y^2-Xy+1=0$\\
Dont le déterminant est :
$\Delta_y=X^2-4$\\
D'où les solutions :\\
$y=\frac{1}{2}(X-\sqrt{X^2-4})$\\
$y=\frac{1}{2}(X+\sqrt{X^2-4})$\\
Cependant, il faut savoir que $\Delta_y$ doit être positif afin que les racines soient rélles.\\

%ICI METTRE DEMO ELIMINATION DE LA SOLUTION X1

%$\Delta>0 \Rightarrow X^2>4$\\
%Pour la solution $X=(-1-\sqrt{1+l_0^2})^2$, nous avons deux cas :
%\begin{itemize}
%\item $(-1-\sqrt{1+l_0^2})^2 > 2$ qui est une condition irréalisable
%\item $(-1-\sqrt{1+l_0^2})^2 < -2 \Rightarrow \sqrt{1+l_0^2}>1$ qui est une condition toujours satisfaite.
%\end{itemize}
%Finalement, il n'y a aucune condition sur $l_0$ pour l'existance de des solutions de $X_1$.\\
Pour la solution $X=(-1+\sqrt{1+l_0^2})^2$, nous avons deux cas de condition d'existance :\\
\begin{itemize}
\item $(-1+\sqrt{1+l_0^2})^2 > 2 \Rightarrow \sqrt{1+l_0^2}>3\Rightarrow l_0 > \sqrt{8}$
\item $(-1+\sqrt{1+l_0^2})^2 < -2 \Rightarrow \sqrt{1+l_0^2}<-1$ qu iest une condition irréalisable
\end{itemize}
Finalement, on déduit que la condition d'existence des points d'équilibres est $l_0>\sqrt{8}$.
On peut alors trouver l'expression des ordonnées des points d'équilibre $y_1,y_2,y_3,y_4$ en fonction de $l_0$ :\\
\begin{itemize}
\item $y_0=0$\\
(Déterminé précédemment)\\
\item $y_1=\frac{1}{2}(X_1-\sqrt{X_1-4})=\frac{1}{2}(-1-\sqrt{1+l_0^2})-\frac{1}{2}(\sqrt{(-1-\sqrt{1+l_0^2})^2-4})$\\
%$=\frac{1}{2}(-1-\sqrt{1+l_0^2})-\frac{1}{2}\sqrt{1+2\sqrt{1+l_0^2}+(1+l_0^2)-4}$\\
%$=\frac{1}{2}(-1-\sqrt{1+l_0^2})-\frac{1}{2}\sqrt{-2+2\sqrt{1+l_0^2}+l_0^2}$\\
Cette solution est strictement négative alors que y(t) est positive, selon la rampe $y=\frac{x^2}{2}$. On doit alors l'éliminer.\\
\item $y_2=\frac{X_1+\sqrt{X_1-4}}{2}=\frac{1}{2}(-1-\sqrt{1+l_0^2})+\frac{1}{2}\sqrt{-2+2\sqrt{1+l_0^2}+l_0^2}$\\
Cette solution est aussi à éliminer. En effet, nous avons : \\
$y_2 >0 \Rightarrow (-1-\sqrt{1+l_0^2}+\sqrt{(-1-\sqrt{1+l_0^2})^2+l_0^2} > 0$\\$\Rightarrow \sqrt{(-1-\sqrt{1+l_0^2})^2+l_0^2}>1+\sqrt{1+l_0^2}$\\$ \Rightarrow 0>\sqrt{1+l_0^2}$ ce qui est impossible.
\item $y_3=\frac{X_2-\sqrt{X_2-4}}{2}=\frac{1}{2}(-1+\sqrt{1+l_0^2})-\frac{1}{2}\sqrt{(-1+\sqrt{1+l_0^2})^2-4}$\\
%$=\frac{1}{2}(-1+\sqrt{1+l_0^2})-\frac{1}{2}\sqrt{1-2\sqrt{1+l_0^2}+(1+l_0^2)-4}$\\
$=\frac{1}{2}(-1+\sqrt{1+l_0^2})-\frac{1}{2}\sqrt{-2-2\sqrt{1+l_0^2}+l_0^2}$\\
\item $y_4=\frac{1}{2}(-1+\sqrt{1+l_0^2})+\frac{1}{2}\sqrt{(-1+\sqrt{1+l_0^2})^2-4}$\\
$=\frac{1}{2}(-1+\sqrt{1+l_0^2})+\frac{1}{2}\sqrt{-2-2\sqrt{1+l_0^2}+l_0^2}$\\
\end{itemize}
Seules deux solutions sont alors retenues, l'unique condition d'existance étant $l_0>\sqrt{8}$.\\
En considérant $y=\frac{x^2}{2} \Longrightarrow x=\pm \sqrt{2y}$ on a au total 5 points d'équilibre :\\
$x_0=0$\\
$x_1=\sqrt{-1+\sqrt{1+l_0^2}-\sqrt{-2-2\sqrt{1+l_0^2}+l_0^2}})$\\
$x_2=-\sqrt{-1+\sqrt{1+l_0^2}-\sqrt{-2-2\sqrt{1+l_0^2}+l_0^2}})$\\
$x_3=\sqrt{-1+\sqrt{1+l_0^2}+\sqrt{-2-2\sqrt{1+l_0^2}+l_0^2}}$\\
$x_4=-\sqrt{-1+\sqrt{1+l_0^2}+\sqrt{-2-2\sqrt{1+l_0^2}+l_0^2}}$\\
Il reste à déterminer la nature de ces points d'équilibre du système.

\subsection{Détermination de la nature des points d'équilibre}
Nous sommes dans le cas d'un système non linéaire.
On pose $x_1=x, x_2=\dot{x}$\\
On a alors :
\[\dot{x_1}=x_2=f_1(x_1*,x_2*)\]
\[ \dot{x_2}=-\frac{x^2_2.x_1}{1+x^2_1} - \frac{x^3_1}{2(1+x^2_1)} - \frac{x_1}{1+x^2_1} + \frac{ax^3_1}{2(1+x^2_1)(x^4_1/4 + 1)}=f_2(x_1,x_2) \]



Aux points d'équilibre, la vitesse est nulle d'où $x_2=0$ et $\dot{x_2}=0$\\
D'où :\\
\[ f_1(x_1*,0) = 0 \]
\[ f_2(x_1*,0) = \frac{x_1^3\times l_0}{2\sqrt{1+\frac{x_1^4}{4}}(1+x_1^2)}-\frac{x_1^3}{2(1+x_1^2)} - \frac{x}{1+x^2} \]

On va calculer la nature des points d'équilibre en passant par la matrice Jacobienne :\\
\[
J=
\begin{pmatrix}
\frac{\delta f_1}{dx_1} & \frac{\delta f_1}{dx_2} \\
\frac{\delta f_2}{dx_1} & \frac{\delta f_2}{dx_2}
\end{pmatrix}
=
\begin{pmatrix}
0&1\\
\frac{\delta f_2}{dx_1}&0
\end{pmatrix}
\]

La nature des points d'équilibre du système est définie par la trace, le déterminant, et le discriminant du polynôme caractéristique de la matrice Jacobienne ci-dessus.\\
$ Tr(J) = 0 $\\
$ det(J) = -\frac{\delta f_2}{\delta x_1} $\\
$ \Delta(J) = Tr^2(J) - 4det(J) = 4*frac{\delta f_2}{\delta x_1} $\\

On a alors besoin de déterminer le signe de $frac{\delta f_2}{\delta x_1}$ aux points d'équilibre.\\

En factorisant $f_2$, on a : \\
$f_2=0=-\frac{x}{2(1+x^2)}*(x^2-\frac{x^2\times l_0}{\sqrt{1+\frac{x^4}{4}}}+2)$\\
car $\ddot(x)=0$ aux points d'équilibre.\\

Ici, nous avons deux éventualités pour satisfaire cette équation :\\
\begin{itemize}
\item cas 1 : $\frac{x}{1+x^2}=0$\\
\item cas 2 : $x^2-\frac{x^2\times l_0}{\sqrt{1+\frac{x^4}{4}}}+2=0$\\
\end{itemize}

\subsubsection*{Premier Cas}
$\frac{x}{1+x^2}=0 \longrightarrow x=0$\\
On a alors :\\
$frac{\delta f_2}{\delta x_1}=\frac{-1}{2}\frac{\delta \frac{x}{1+x^2}}{\delta x}(x^2 - \frac{x^2\times l_0}{\sqrt{1+\frac{x^4}{4}}}+2)=-2\frac{1+x^2-4x^2}{2(1+x^2)^2}$
Ce qui équivaut, avec l'hypothèse précédente $x=0$, à :\\
\[frac{\delta f_2}{\delta x_1}=-1\]

La matrice jacobienne du système pour le point d'équilibre $x=0$ devient donc :\\
\[
J=
\begin{pmatrix}
0&1\\
-1&0
\end{pmatrix}
\]
On peut alors en calculer les caractéristique :\\
$tr(J_f) = 0 $\\
$det(J_f) =  1 $\\
$\Delta(J_f) = -4$\\
Ce sont les caractéristiques d'un point d'équilibre centré.\\

\subsubsection*{Second Cas}
Dans ce cas, $x^2-\frac{x^2\times l_0}{\sqrt{1+\frac{x^4}{4}}}+2=0$\\
On a alors :\\
$\frac{\delta f_2}{\delta x}=-\frac{x}{2(1+x^2)}(2x-\frac{2x\times l_0\sqrt{1+\frac{x^4}{4}}-x^2\times l_0\frac{x^3}{2\sqrt{1+\frac{x^4}{4}}}}{1+\frac{x^4}{4}})=
-\frac{x}{2(1+x^2)}(2-\frac{2\times l_0}{\sqrt{1+\frac{x^4}{4}}}+l_0\frac{x^4}{2\sqrt{1+\frac{x^4}{4}^{3/2}}})$\\$= \frac{-x^2}{2(1+x^2)}\frac{2(1+\frac{x^4}{4})-\frac{2}{x^2}(2+x^2)(1+\frac{x^4}{4})+\frac{x^2}{2}(2+x^2)}{1+\frac{x^4}{4}}=\frac{-x^2+2}{(1+x^2)(1+\frac{x^4}{4})}=\frac{(x-\sqrt{2})(x+\sqrt{2})}{(1+x^2)(1+\frac{x^4}{4})}$\\
Etudier le signe de $\frac{\delta f_2}{\delta x}$ revient donc à étudier le signe de $-x^2+2$.\\
On en calcule alors le discriminant pour connaitre les solutions pour dresser, par la suite, un tableau de variation.
$\Delta = 8$\\
D'où les solutions :\\
$x_1 = -\frac{\sqrt{8}}{-2} = -\sqrt{2}$\\
$x_2 = \frac{\sqrt{8}}{-2} = \sqrt{2}$\\
D'où le tableau de variations suivant :\\
\begin{tikzpicture}
\tkzTab{$x$/1,$\frac{\delta f_2}{\delta x_1}$/1,ENLEVER CETTE LIGNE/1.8}{$-\infty$,$+\infty$}{+,-,+}{-/$-\infty$, +/$+\infty$}
\tkzTabVal[draw]{1}{2}{0.4}{$-\sqrt{2}$}{$0$}
\tkzTabVal[draw]{1}{2}{0.6}{$\sqrt{2}$}{$0$}\\
\end{tikzpicture}

%%%%%%%%%%%%%%%%%%%%%%%%%%%%
$x=0$\\
$x_1=\sqrt{-1+\sqrt{1+l_0^2}-\sqrt{-2-2\sqrt{1+l_0^2}+l_0^2}})$\\
$x_2=-\sqrt{-1+\sqrt{1+l_0^2}-\sqrt{-2-2\sqrt{1+l_0^2}+l_0^2}})$\\
$x_3=\sqrt{-1+\sqrt{1+l_0^2}+\sqrt{-2-2\sqrt{1+l_0^2}+l_0^2}}$\\
$x_4=-\sqrt{-1+\sqrt{1+l_0^2}+\sqrt{-2-2\sqrt{1+l_0^2}+l_0^2}}$\\
%%%%%%%%%%%%%%%%%%%%%%%%%%%%

On peut alors étudier la nature des poitns d'équilibre.
En partant de la condition d'existence des points d'équilibre:\\
$l_0>2\sqrt{2}$\\
$1+l_0^2>9$\\
$-1+\sqrt{1+l_0^2}>2$\\
avec :\\
$\sqrt{(-1+\sqrt{1+l_0^2})^2-4}>0$\\
Ce qui nous permet enfin de calculer la nature des points d'équilibre :
\begin{itemize}
\item $x_0=0$\\
D'après le tableau de variation, $\frac{\delta f_2}{\delta x}$<0.\\
$tr(J_f) = 0 $\\
$det(J_f) > 0 $\\
$\Delta(J_f) < 0$\\
Il s'agit d'un point centre.
\item $x_1=\sqrt{-1+\sqrt{1+l_0^2}-\sqrt{-2-2\sqrt{1+l_0^2}+l_0^2}})$\\
Correspondant à : $y_3=\frac{1}{2}(-1+\sqrt{1+l_0^2})-\frac{1}{2}\sqrt{(-1+\sqrt{1+l_0^2})^2-4}$\\
On pose :\\
$A=-1+\sqrt{1+l_0^2}$\\
$B=\sqrt{(-1+\sqrt{1+l_0^2})^2-4} = \sqrt{A^2-4}$\\
$y=\frac{1}{2}(A-B)$\\
D'après le théorème des trois maisons, on a :\\
$B > \sqrt{A^2} - \sqrt{4}$\\
$B > A - 2$\\
$A-B < 2$\\
$\frac{1}{2}(A-B)<1$\\
$y_3 < 1$\\
$\frac{x_1^2}{2}<1$\\
$x_1^2<2$\\
D'où :
$x_1<\sqrt{2}$ et $x_1>-\sqrt{2}$\\
D'après le tableau de variations, $\frac{\delta f_2}{\delta x_1}<0$. Ainsi :\\
$tr(J_f) = 0 $\\
$det(J_f) > 0 $\\
$\Delta(J_f) < 0$\\
Il s'agit d'un point centre.
\item $x_2=-\sqrt{-1+\sqrt{1+l_0^2}-\sqrt{-2-2\sqrt{1+l_0^2}+l_0^2}})$\\
Il s'agit de l'opposé de $x_1$, on est donc aussi dans le cas de $\frac{\delta f_2}{\delta x_1}<0$.\\
$x_2$ est donc, comme $x_1$, un point centre.
\item $x_3=\sqrt{-1+\sqrt{1+l_0^2}+\sqrt{-2-2\sqrt{1+l_0^2}+l_0^2}}>\sqrt{2}$\\
Car racine carrée d'une somme de expressions supérieures à zéro, dont l'une supérieur à 2.
D'après le tableau de variation, $\frac{\delta f_2}{\delta x}>0$. Ainsi :\\
$tr(J_f) = 0 $\\
$det(J_f) < 0 $\\
$\Delta(J_f) > 0$\\
Il s'agit d'un point selle.
\item $x_4=-\sqrt{-1+\sqrt{1+l_0^2}+\sqrt{-2-2\sqrt{1+l_0^2}+l_0^2}}<-\sqrt{2}$\\
Puisqu'il s'agit de l'opposé de $x_3$.
D'après le tableau de variation, $\frac{\delta f_2}{\delta x}>0$.\\
$x_4$ est donc, comme $x_3$, un point selle.\\
\end{itemize}
On représente les $x_n$ sur le portrait de phase.
%LA CE SERAIT SYMPA DE FAIRE UN SCHEMA AVEC EN GROS LES Xn PLACÉS A PEU PRES.


%D'après l'équation:\\
%$\ddot{x}=0$\\
%On a :\\
%$\sqrt{1+\frac{x^4}{4}}=\frac{x^2\times l_0}{2+x^2}$\\
%On peut alors simplifier l'équation de la dérivée partielle de $f_2$, pour obtenir :\\
%$\frac{\delta f_2}{\delta x}=-\frac{x}{2(1+x^2)}(2-\frac{2(2+x^2}{x^2}+\frac{x^2}{2}\frac{2+x^2}{1+\frac{x^4}{4}})$
%
%Pour étudier le signe de $\frac{\delta f_2}{\delta x}$, on la compare à 0 :
%$2x^2-2(2+x^2)+\frac{x^4}{4}\frac{2+x^2}{1+\frac{x^4}{4}}=0$\\
%$2x^2-\frac{x^6}{2}-(4+2x^2)(1+\frac{x^4}{4})+x^4+\frac{x^6}{2}=0$\\
%$2x^2+x^6-4-x^4-2x^2-\frac{x^6}{2}+x^4=0$\\
%$\frac{x^6}{2}-4=0$\\
%$x^6=8$\\
%On peut alors déterminer le signe de $\frac{\delta f_2}{\delta x}$ qui est directement lié à celui de $x^6-8$, afin de trouver le signe de :\\
%$tr(J_f) = 0 $\\
%$det(J_f) =  -C\times(x^6-8) $\\
%$\Delta(J_f) = 4C\times(x^6-8)$\\
%Avec C une fonction strictement positive.
%
%Les relations qui lient le signe de $\frac{\delta f_2}{\delta x_1}$ à celui de $x^6-8$ sont les suivantes :\\
%$\frac{\delta f_2}{\delta x_1}<0$ si $x^6<8$\\
%$\frac{\delta f_2}{\delta x_1}=0$ si $x^6=8$\\
%$\frac{\delta f_2}{\delta x_1}>0$ si $x^6>8$\\

%Il faut alors déterminer $\frac{\delta f_2}{dx_1}$ : \\
%\[ \frac{\delta f_2}{dx_1} = \frac{3l_0x_1^2+l_0x^4 - (1+x^2)\frac{l_0x^6}{2(1+\frac{x^4}{4})}}{2\sqrt{1+\frac{x^4}{4}}(1+x^2)^2}-\frac{3x^2+x^4}{2(1+x^2)^2} \]
%Détail du calcul de  $\frac{\delta f_2}{dx_1}$ :\\
%(VIDE !!! voir lune à plume, je vois de quoi je parle. Pour info, c'est du calcul brut, pas "d'astuce")\\
%
%On peut alors calculer les propriétés de la matrice (respectivement Déterminant, Trace, Discriminant du polynôme caractéristique) :\\
%\[ det(J_f) =  \frac{\delta f_2}{dx_1} \]
%\[ tr(J_f) = 0 \]
%\[ \Delta(J_f) = tr(J_f)^2 - 4\times det(J_f) = 4\frac{\delta f_2}{dx_1}\]
%Pour connaitre la nature du point d'équilibre, il nous reste plus qu'à étudier le signe de $\frac{\delta f_2}{dx_1}$ :\\
%\[ \frac{\delta f_2}{dx_1} = \frac{3l_0x_1^2+l_0x^4 - (1+x^2)\frac{l_0x^6}{2(1+\frac{x^4}{4})}}{2\sqrt{1+\frac{x^4}{4}}(1+x^2)^2}-\frac{3x^2+x^4}{2(1+x^2)^2} \]
%D'après la question précédente, on a en tout point d'équilibre $\sqrt{1+\frac{x^4}{4}}=\frac{x^2l_0}{x^2+2}$, d'où : \\
%\[ \frac{\delta f_2}{dx_1} = \frac{3l_0x_1^2+l_0x^4 - \frac{1}{2}(x^2+2)^2(1+x^2)\frac{l_0x^6}{\frac{(x^2l_0)^2}{(x^2+2)}}}{2(1+x^2)^2}-\frac{3x^2+x^4}{2(1+x^2)^2} \]
%\[ =\frac{-\frac{1}{2l_0}x^2(x^2+2)^2(1+x^2)(x^2+2)+(3l_0x^2+l_0x^2+l_0x^4)(x^2+2)-3x^2-x^4}{2(1+x^2)^2} \]
%
%(Jusqu'à là, je suis sur que c'était la bonne méthode. Le remplacement de $\sqrt{1+\frac{x^4}{4}}=\frac{x^2l_0}{x^2+2}$ est lui-aussi bon) \\
%
%Le dénominateur étant positif pour tout x, on cherche le signe du numérateur, c-à-d de :\\
%\[ -\frac{1}{2l_0}x^2(x^2+2)^2(1+x^2)(x^2+2)+(3l_0x^2+l_0x^2+l_0x^4)(x^2+2)-3x^2-x^4 \]
%
%%Thibault:
%Étude de signe du polynôme:\\
%on pose, \[ -\frac{1}{2l_0}x^2(x^2+2)^2(1+x^2)(x^2+2)+(3l_0x^2+l_0x^2+l_0x^4)(x^2+2)-3x^2-x^4  = 0\]
%
%Le polynôme est factorisable par $x^2$ et on obtient un polynôme d'ordre 8\\
%$-1x^8/l_0-7x^6/l_0-18x^4/l_0+x^2(n 6l_0-1-20/l_0)+(8l_0-1-8l_0) = 0$\\
%On pose :\\ $V = -1/l_0$\\   $Y = (6l_0-1-20/l_0)$\\   $Z = (8l_0-1-8/l_0)$\\
%on a: $Vx^8+7Vx^6+18Vx^4+Yx^2+Z = 0$\\
%le polynôme est d'ordre 8 c'est donc un ordre pair donc il admet au maximum deux racines.\\
%Or le coefficient de l'ordre 8 est impair et z est négatif donc le polynôme admet deux racines de signes contraires.\\
%on peut donc factoriser le polynôme par un polynôme d'ordre 6 et par un polynôme d'ordre 2\\










\section{On suppose que $a=\sqrt{15}$.}
\subsection{Déterminer la valeur exacte des points d'équilibres du système.}
On détermine les valeurs numériques des points d'équilibres pour $a=\sqrt{15}$. On a:\\
$x_0=0$\\
$x_1=\sqrt{3 - \sqrt{5}}=0.874$\\
$x_2=\sqrt{3 - \sqrt{5}}=-0.874$\\
$x_3=-\sqrt{3 + \sqrt{5}}=2.288$\\
$x_4=-\sqrt{3 + \sqrt{5}}=-2.288$\\

\subsection{Déterminer l'intégrale première du système.}
	On rappelle que $v=\dot{x}\sqrt{1 + x^2}$ et $\frac{\delta{v}}{\delta{t}}=\ddot{x}\sqrt{1 + x^2} + \frac{\dot{x}^2.x}{\sqrt{1+x^2}}$ \\
On va intégrer cette équation différentielle.
\[  \ddot{x}+\frac{\dot{x}^2x}{1+x^2} + \frac{1}{2}\frac{x^3}{1+x^2} + \frac{x}{1+x^2} - \frac{a.x^3}{2(1+x^2)\sqrt{x^4/4+1}}=0 \]	
On multiplie cette équation par $\sqrt{1+x^2}$:

\[  \ddot{x}\sqrt{1+x^2}+\frac{\dot{x}^2x}{\sqrt{1+x^2}} + \frac{1}{2}\frac{x^3}{\sqrt{1+x^2}} + \frac{x}{\sqrt{1+x^2}} - \frac{a.x^3}{2\sqrt{1+x^2}\sqrt{x^4/4+1}}=0 \]	

On remarque ainsi que l'équation s'écrit de la forme:\\
\[ \frac{\delta{v}}{\delta{t}} + \sqrt{1+x^2}f(x) = 0 \] avec $f(x)=  \frac{1}{2}\frac{x^3}{1+x^2} + \frac{x}{1+x^2} - \frac{a.x^3}{2(1+x^2)\sqrt{x^4/4+1}}$ \\
On multiplie cette équation par v. On a alors:\\
$ v.\frac{\delta{v}}{\delta{t}} + \dot{x}(1+x^2)f(x) = 0$\\
En intégrant l'équation, on a alors:\\
$ \int v.\frac{\delta{v}}{\delta{t}} + \dot{x}(1+x^2)f(x) dt= C$\\
Ce qui revient à écrire:\\
$ \int vdv + \int (1+x^2)f(x)dx= C$\\
En intégrant l'équation, on obtient:\\
%\[\fbox{$ \frac{1}{2}\dot{x}^2(1+x^2) - \frac{1}{8}x^4 - \frac{x^2}{2} + a\sqrt{\frac{x^4}{4} + 1} = C  $}\]
\[\fbox{$ \frac{1}{2}v^2 + \frac{1}{8}x^4 + \frac{x^2}{2} - a\sqrt{\frac{x^4}{4} + 1} = C  $}\]

 

	
\subsection{Représenter le portrait de phase.}
A partir de l'intégrale première, on détermine le portrait de phase pour $a=\sqrt{15}$.
\begin{figure}[H]
	\includegraphics[scale=0.5]{PortraitDePhasePointsVisibles.png}
	\caption{Portrait de phase. $a=\sqrt{15}$}
\end{figure} 

\subsection{Que peut-on en déduire sur le mouvement.}
On remarque trois points d'équilibres en (0,0), (2.288,0) et (-2.288,0). En ces points le mouvement tend à s'arrêter. On remarque aussi deux points selles en (-0.874,0) et (0.874,0). En ces points le système est instable et le mouvement tend à s'accélérer.%Il faudra pe modifier mon analyse.

\section{On suppose maintenant que $a=\sqrt{3}$ et $x(0)=x_0>0$ et $\dot{x}(0)=0$.}
\subsection{Calculer et représenter à l'aide de Matlab la période T en fonction de $x_0$ pour $0<x_0<10$.}
On peut déterminer la période T des oscillations en calculant cette intégrale:\\
$T=2\int_{t(x_{min})}^{t(x_{max})} dt $\\
\\
L'intégrale première du système peut s'écrire sous la forme: $ \frac{v^2}{2} + G(x) = 0 $ avec $ G(x)=\frac{1}{8}x^4 + \frac{x^2}{2} - a\sqrt{\frac{x^4}{4} + 1}  $\\
C correspond à la valeur initiale: $C=G(x_0)$ pour $x_0>0$\\
Puisque $ \frac{1}{2}\dot{x}^2(1 + x^2) + G(x) = G(x_0) $. On a alors $\dot{x} = \sqrt{\frac{2(G(x_0) - G(x))}{1 + x^2}}$\\
Connaissant $\frac{\delta x}{\delta t}$, on peut simplifier le calcul de l'intégrale pour la période T:\\
$T=2\int_{x_{min}}^{x_{max}} \frac{\sqrt{1 + x^2}}{\sqrt{2(G(x_0) - G(x))}} dx $\\
On intègre de 0 à $x_0$. On a alors:
\[T=2\int_{0}^{x_0} \frac{\sqrt{1 + x^2}}{\sqrt{2(G(x_0) - G(x))}} dx \]


\section{On suppose maintenant que le système est soumis à une force de frottement $\gamma > 0$ et que l'équation devient: (E) $\ddot{x} + \gamma.\dot{x} + f(x,a) = 0$.}
\subsection{Représenter le diagramme de Matlab le diagramme de bifurcation  en $(a,\gamma)$ pour chacun des points d'équilibres.}
L'équation différentielle s'écrit désormais:\\
\[  \ddot{x} + \gamma.\dot{x} + \frac{\dot{x}^2x}{1+x^2} + \frac{1}{2}\frac{x^3}{1+x^2} + \frac{x}{1+x^2} - \frac{a.x^3}{2(1+x^2)\sqrt{x^4/4+1}}=0 \]	
%On détermine les points d'équilibres: Les points d'équilibres sont les memes car \ddot{x}=\dot{x}=0
On a toujours:\\
$x_0=0$\\
$x_1=\sqrt{-1+\sqrt{1+l_0^2}-\sqrt{-2-2\sqrt{1+l_0^2}+l_0^2}})$\\
$x_2=-\sqrt{-1+\sqrt{1+l_0^2}-\sqrt{-2-2\sqrt{1+l_0^2}+l_0^2}})$\\
$x_3=\sqrt{-1+\sqrt{1+l_0^2}+\sqrt{-2-2\sqrt{1+l_0^2}+l_0^2}}$\\
$x_4=-\sqrt{-1+\sqrt{1+l_0^2}+\sqrt{-2-2\sqrt{1+l_0^2}+l_0^2}}$\\

Il faut calculer pour chaque points d'équilibres:\\
D= Calcul du discriminant du polynome caractéristique\\
S= Somme des 2 valeurs propres\\
P= Produit des 2 valeurs propres \\

Pour cela, on étudie la matrice du système linéaire associée.

\[
J=
\begin{pmatrix}
\frac{\delta f_1}{\delta x_1} & \frac{\delta f_1}{\delta x_2}\\
\frac{\delta f_2}{\delta x_1} & \frac{\delta f_2}{\delta x_2} \\
\end{pmatrix}
\]

On a toujours \[\dot{x_1}=x_2=f_1(x_1*,x_2*)\]
D'où $\frac{\delta f_1}{\delta x_1}=0$ et $\frac{\delta f_1}{\delta x_2}=1$\\
Le terme supplémentaire n'a pas une dérivée nulle par rapport à $x_2$: $\frac{\delta \gamma x_2}{\delta x_2}=\gamma$\\.
Nous obtennons ainsi la nouvelle matrice Jacobienne associée au système avec forces de frottements :
\[
J=
\begin{pmatrix}
0 & 1\\
\frac{\delta f_2}{\delta x_1} & \gamma\\
\end{pmatrix}
\]
\\
$tr(J) = \gamma $\\
$det(J) = -\frac{\delta f_2}{\delta x_1} $\\
$\Delta(J) = tr^2(J)-4det(J)=\delta^2 + 4\frac{\delta f_2}{\delta x_1}$\\



\subsection{On suppose que $a=\sqrt{15}$. Pour quelles valeurs (exactes) de $\gamma$ les points d'équilibres attractifs changent-ils de nature.}

\subsection{Représenter le portrait de phase pour $\gamma=1, \gamma=2, \gamma=3.$}
On calcul l'intégrale première du système afin de pouvoir déterminer le diagramme de phase. On a:
\[\fbox{$ \frac{1}{2}v^2 + \gamma.v + \frac{1}{8}x^4 + \frac{x^2}{2} - a\sqrt{\frac{x^4}{4} + 1} = C  $}\]
IL FAUT UTILISER LA METHODE 2 (local)
%Il y a plus qu'a tester sur matlab.
%gamma=1
%gamma=2 
%gamma=3
\end{document}